%--------------------------------------------------------------------
\medskip
\section{Flocking}
En esta sección va a tratar sobre la implementación del Flocking pedido en el proyecto. Todo lo referente a esta sección se encuentra dentro del paquete \texttt{com.mygdx.iadevproject.aiReactive.flocking}, en la carpeta de test. \\

Debido a que la finalidad del videojuego no es muy idílica para incluir un flocking en él, hemos decidido implementar el flocking a parte (de ahí que esté en el apartado de test, y no en la parte de código fuente). \\

Para conseguir este comportamiento, hemos utilizado un árbitro por mezcla ponderada y donde todos los personajes tienen los siguientes comportamientos:
\begin{itemize}
 \item \texttt{CollisionAvoidance} para evitar choques.
 \item \texttt{Separation} para separase del grupo.
 \item \texttt{Cohesion} para juntarse al grupo.
 \item \texttt{VelocityMatching} para ajustarse a la velocidad del personaje que guía al flocking.
 \item \texttt{LookingWhereYouGoing} para que miren por dónde van.
 \item \texttt{Wander} para que parezca que hacen cosas aleatorias.
\end{itemize}

Cada uno de estos comportamientos tienen un peso distinto, por ejemplo, el \texttt{CollisionAvoidance} es el que tiene mayor peso, mientras que el \texttt{Wander} es el que tiene menor peso. Para guiar a todo el grupo, se ha creado un personaje que simplemente realiza un \texttt{Arrive} a la posición que se clicka en pantalla. Todos los demás personajes hacen el \texttt{VelocityMatching} a este personaje. \\

El funcionamiento del test es el siguiente: hay cinco personajes (cubos) que son a los que se aplica el Flocking y otro personaje (gota) que es al que el grupo de cubos hace el \texttt{VelocityMatching}. De esta manera, el usuario puede clickar en la pantalla y la gota lo que hará será ir hacia esa posición realizando un \texttt{Arrive}, lo cual hará que los cubos empiecen a ir en esa dirección. Hay que tener en cuenta que la gota \textbf{simplemente sirve para modificar la dirección de la velocidad del grupo}; es decir, \textbf{el Flocking no se aplica a la gota}.
