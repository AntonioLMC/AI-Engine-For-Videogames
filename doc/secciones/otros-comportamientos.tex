%--------------------------------------------------------------------
\medskip
\section{Otros comportamientos}

A parte de todos los comportamientos descritos anteriormente, para la parte táctica también se han implementado algunos otros comportamientos. Realmente, no son comportamientos como tal, sino que son diversas acciones o procesos para los cuales ha resultado conveniente el poder tratarlos homogéneamente y como si fueran comportamientos normales. Estas acciones son el \textbf{ataque} y la \textbf{cura}. Cabe destacar que estos comportamientos \textbf{devolverán el steering nulo}, puesto que, realmente, lo importante de estos comportamientos no es el comportamiento propiamente dicho, sino todas las funciones/comprobaciones que se realizan dentro de él (en el método \texttt{getSteering}).

\subsection{Comportamiento de ataque}

Este comportamiento representa la orden o la acción de atacar a un objetivo. Por tanto, los elementos que este comportamiento necesita como entrada son: el personaje que realiza el ataque, el personaje que recibe el ataque, el daño que realiza el personaje origen al personaje destino (salud que se le resta al personaje destino) y la máxima distancia a la que se puede realizar el ataque. \\

En primer lugar, se comprueba si la distancia a la que se encuentran origen y destino es menor o igual que la distancia máxima permitida para realizar el ataque. En caso afirmativo, se llamará al método \texttt{reduceHealth} de la clase \texttt{Character}, para reducir la vida al target. \\

Cuando un personaje (no formación) realiza un ataque, este comportamiento será añadido a su lista de comportamiento. Tal y como se puede observar, el hecho que tratar el mecanismo de ataque como un comportamiento más, nos permite poder añadirlo a la lista de comportamientos de un personaje y poder tratarlo de manera homogénea con el resto de comportamientos.

\subsection{Comportamiento de cura}

Este comportamiento representa el proceso de cura (incremento de salud) por parte de un personaje. Por tanto, los elementos que este comportamiento necesita como entrada son: el personaje fuente cuya salud va a ser incrementada y el valor de salud a incrementar. \\

En este caso, lo único que se hace en el método \texttt{getSteering} es llamar al método \texttt{addHealth} de la clase \texttt{Character}. Con esto conseguimos incrementar el nivel de salud de un personaje fuente. \\

Al igual que antes, cuando un personaje (no formación) se cura, este comportamiento será añadido a su lista de comportamientos.

\subsection{Ataque y cura en las formaciones}

A la hora de entender los procesos de ataque y cura para el caso de las formaciones, hay que tener muy claro lo siguiente:
\begin{itemize}
	\item Cuando un conjunto de personajes están en una formación, la lista de comportamientos propia de cada personaje no se tiene en cuenta. Por el contrario, se usará la lista de comportamiento del propio objeto formación (para mover el ancla de la formación) y un conjunto de comportamientos establecidos directamente ``a pelo'' en el método correspondiente para hacer que los componentes de la formación vayan al punto que les corresponde.
	\item Los objetos formación como tal \textbf{ni atacan ni se curan}. Los que atacan son los elementos de la formación.
	\item Del mismo modo, los objetos de tipo formación no pueden ser atacados (a la hora de atacar se comprueba que el objetivo no sea una formación) ni se pueden curar (ya que, de nuevo, los que se curan son los componentes de la formación).
	\item Los componentes de la formación no tienen estructura táctica (máquina de estados o árbol de decisión). Esta estructura solamente se encuentra en el objeto tipo formación (en la ``raíz'', en caso de haber formaciones de formaciones).
\end{itemize}

Sabiendo todo esto, es necesario contar con algún mecanismo para el objeto tipo formación (el ancla) pueda comunicar a los integrantes de la formación que deben curarse o atacar a un objetivo determinado. Esta comunicación se realiza a través de los métodos \texttt{enableAttackMode} y \texttt{disableAttackMode} (para el caso del comportamiento de ataque) y los métodos \texttt{enableCure} y \texttt{disableCure} (estos métodos se encuentran en la clase \texttt{Formation}). Al llamar a estos métodos lo que hacemos es modificar los flags correspondientes y almacenar otros atributos necesarios para los comportamientos de ataque y cura. Teniendo en cuenta esos flags, en el método \texttt{getComponentFormationSteerginToApply} de cada tipo de formación concreta se añadirá el comportamiento de ataque o cura a la lista de comportamientos establecida ``a pelo'' que, como sabemos, se usará para controlar a cada uno de los componentes de la formación. De esta manera conseguimos el poder imponer el comportamiento de ataque o cura a cada uno de los elementos que componen una formación.

