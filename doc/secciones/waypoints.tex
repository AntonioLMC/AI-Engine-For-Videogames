%--------------------------------------------------------------------
\medskip
\section{Waypoints}
\label{waypoints}

Un Waypoint es una posición preestablecida en el mapa que puede tener cierta finalidad y puede ser usada para un determinado propósito. Estos puntos en el mapa normalmente son establecidos por los propios diseñadores del mapa y pueden tener diversos usos como uso táctico (para este caso, ocupar estas posiciones proporciona algún tipo de ventaja táctica) o pueden representar posiciones defensivas (posiciones más adecuadas para situar personajes defensivos). En cualquier caso, estos puntos deben tener asociada cierta información que dependerá del uso que les queramos dar. \\

Para nuestro caso concreto, vamos a implementar y a usar waypoints con finalidad defensiva, es decir, los waypoints establecidos representarán posiciones interesantes a defender. Toda esta funcionalidad se encuentra en el fichero \texttt{Waypoints} del paquete \texttt{com.mygdx.iadevproject.waypoints}. \\

Cada equipo tendrá 2 conjuntos de waypoints:
\begin{itemize}
	\item Waypoints de la base $\rightarrow$ Estos waypoints serán usados para defender la base. Los personajes cuyo rol sea soldado defensivo harán un Pathfollowing a la lista de puntos correspondientes a su base.
	\item Waypoints de los puentes $\rightarrow$ Cada equipo dispone de 6 waypoints de este tipo. Tal y como se ha explicado en el apartado correspondiente, en el mapa hay 4 puentes, aunque solo 3 de ellos serán defendidos por los personajes (cuyo rol sea arquero defensivo) de cada equipo. Para cada puente de esos 3, un equipo tendrá 2 waypoints en su lado correspondiente del puente (lo que harán un total de 6 waypoints para cada equipo). Al igual que antes un personaje aplicará un Pathfollowing para patrullar un lado de uno de los puentes (Pathfollowing con una lista de 2 puntos).
\end{itemize}

Para repartir de manera ordenada los waypoints de los puentes, se ha implementado un sistema de reserva y liberación de waypoints (solamente para los waypoints de los puentes). Como solamente hay 6 waypoints en los puentes en cada equipo, solamente podrá haber 6 arqueros defensivos patrullando los puentes. Si otros personajes intentan reservar un waypoint de los puentes, no podrán y pasarán a moverse de manera aleatoria (mediante el comportamiento Wander). \\

En la clase correspondiente y para el caso de los waypoints de los puentes, podemos encontrar las siguientes estructuras:
\begin{itemize}
	\item \texttt{bridges_CharacterAndWaypointAssociation_team_X} (una para cada equipo) $\rightarrow$ Esta estructura es de tipo \texttt{Map<Character, Vector3>}
	\item \texttt{bridgesWayPoints_team_X} (una para cada equipo) $\rightarrow$ Esta estructura es de tipo \texttt{Map<Vector3, ValueOfBridgeWaypoint>}, se inicializa al principio de la aplicación y 
\end{itemize}

