%--------------------------------------------------------------------
\medskip
\section{Conclusiones}

Este proyecto ha consistido en implementar una serie de elementos de inteligencia artificial que se usan muy comúnmente en los videojuegos reales. Además, la propia implementación de dichos elementos se ha llevado a cabo en el contexto de un juego de guerra en tiempo real. Tal y como se ha podido apreciar a lo largo de la documentación, esos elementos van desde la parte reactiva (los comportamientos más básicos, los movimientos de un punto origen a un punto destino y las estructuras que permiten elegir o combinar el comportamiento o comportamientos a aplicar) hasta la parte táctica (que se ha abordado mediante ciertas estructuras de comportamiento táctico y toma de decisiones, como máquina de estados o árboles de comportamiento), pasando además por otras técnicas interesantes que también se usan hoy día en muchos videojuegos reales (mapas de influencia o waypoints). Así mismo, también se ha diseñado e implementado otra funcionalidad para permitir agrupar a los personajes y gestionarlos de manera compacta (formaciones de personajes). \\

El estudio, aplicación e implementación de estas técnicas nos ha permitido comprender de una forma más profunda y mejor el funcionamiento interno de los videojuego reales y qué está pasando realmente dentro de ellos cuando vemos determinadas acciones y comportamientos en una partida. Así mismo, también hemos podido reforzar y coger experiencia con algunos conceptos y técnicas muy interesantes que, muy posiblemente, puedan tener una aplicación práctica en otros ámbitos y no solamente en el mundo de los videojuegos. A parte de los anterior, los conceptos estudiados en la asignatura y la realización de este proyecto también nos ha permitido estudiar y aprender la estructura general de un videojuego, cómo éste se organiza y cómo las distintas partes que lo componen interactúan y se comunican entre sí. \\

Por todo lo anterior, consideramos que estas prácticas han sido muy adecuadas e interesantes, nos han permitido introducirnos (al menos, de forma básica) en el mundo de los videojuegos y en la implementación de éstos y nos han permitido comprender y entender la estructura y organización general de un videojuego.

