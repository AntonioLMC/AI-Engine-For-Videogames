%--------------------------------------------------------------------
\medskip
\section{Steerings}

En el ámbito de los videojuegos, un sistema steering (o sistema de dirección, es español) es un mecanismo que \textit{propone movimientos} a los agentes involucrados en el videojuego en base al entorno local que les rodea, es decir, usando la información del mundo que los agentes captan a través de sus sentidos. Los steerings concretos que se han estudiado y que, por tanto, han sido implementados es este proyecto son: \textbf{Steering Uniforme} y \textbf{Steering Uniformemente Acelerado}. \\

Un Steering Uniforme solamente maneja velocidad lineal y velocidad angular, puesto que se da por hecho que en el movimiento no va a intervenir ningún tipo de aceleración (en decir, la aceleración es 0). A partir de dicha información contenida en el steering, el agente modificará su posición y orientación. Por el contrario, un Steering Uniformemente Acelerado sí contiene una aceleración lineal y una aceleración angular, puesto que en este tipo de movimientos sí intervienen aceleraciones (aceleración constante). A partir de esta información contenida en el steering, el agente modificará su velocidad lineal y angular. Al modificar la velocidad lineal y angular, la posición y orientación del agente también será modificada en consecuencia. 

\subsection{Interfaz Steering}

Durante el desarrollo del proyecto, nos ha parecido conveniente el poder manejar todos los steerings de manera uniforme independientemente de la clase de steering concreto. Esto ha sido posible gracias a ciertas características del lenguaje Java como son las clases abstractas, la herencia o la ligadura dinámica. Teniendo esto en cuenta, hemos implementado una clase abstracta vacía (clase \texttt{Steering}), tal que todos los steerings concretos heredarán de ella.

\subsection{Steering Uniforme}

Este tipo de steering ha sido implementado en la clase \texttt{Steering\_NoAcceleratedUnifMov} y, tal y como se ha comentado, hereda de la clase abstracta \texttt{Steering}. En esta clase podemos encontrar dos atributos llamados \texttt{velocity} (de tipo \texttt{Vector3}) y \texttt{rotation} (de tipo float). El primero hace referencia a la velocidad lineal (el vector velocidad) y el segundo hace referencia a la velocidad angular (un escalar). En esta clase también están presentes los métodos \textit{get} y \textit{set} correspondientes a ambos atributos. Además de estos métodos, también hay otro método denominado \texttt{getSpeed}, que devuelve el módulo del vector \texttt{velocity}. \\

Es muy importante remarcar que, es este ámbito, \textit{velocity} y \textit{speed} no significan lo mismo. El primero hace referencia al \textbf{vector} velocidad, mientras que el segundo hace referencia al \textbf{módulo} de dicho vector (es un escalar).

\subsection{Steering Uniformemente Acelerado}

Este tipo de steering ha sido implementado en la clase \texttt{Steering\_AcceleratedUnifMov} y, tal y como se ha comentado, hereda de la clase abstracta \texttt{Steering}. Esta clase tiene dos atributos llamados \texttt{lineal} (de tipo \texttt{Vector3}) y \texttt{angular} (de tipo \texttt{float}). El primero corresponde con la aceleración lineal (un vector) y el segundo con la aceleración angular (un escalar). También se han implementado los métodos \textit{get} y \textit{set} correspondientes a ambos atributos.

