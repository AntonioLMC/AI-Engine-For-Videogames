%--------------------------------------------------------------------
\medskip
\section{Manual de uso}
En esta sección se va a exponer un breve manual de uso para el usuario. \\

Por defecto, al arrancar el programa, el modo en el que se inicia el juego es el modo de ``batalla final''; esto es, ambos equipos intentan conseguir la victoria, además de evitar, también, la derrota. Así pues, se verá cómo los personajes defensivos tratan de defender su zona, mientras que los personajes ofensivos tratan de ir hacia la base enemiga. \\

Nuestro juego esta configurado inicialmente para el modo batalla final, aunque, se pueden crear y establecer otros muchos modos de juego solamente modificando el comportamiento táctico de los personajes (su rol). Con estas sencillas modificaciones en el código fuente (en la parte de creación de los personajes) se pueden conseguir otros muy distintos y variados sistemas de juego. A parte de todo esto, durante la propia partida se da cierta libertad al usuario para que pueda realizar ciertas acciones con los personajes. Para ello, tiene que realizar lo siguiente:

\begin{itemize}
 \item \textbf{Seleccionar} el personaje (o los \textbf{personajes}) a los que quiera aplicar acciones. Para seleccionar un personaje simplemente se ha de clickar sobre él con el botón izquierdo del ratón. Si se quiere seleccionar varios personajes, se debe mantener pulsada la tecla \texttt{Control-Izquierdo}, de lo contrario, solamente se seleccionará uno. Es importante destacar que solamente se pueden seleccionar varios personajes si estos pertenecen al mismo equipo. Cuando un personaje es seleccionado, dejará de hacer lo que estaba haciendo para realizar todo lo que le mande el usuario.
 
 \item Para \textbf{liberar} a los \textbf{personajes} simplemente se ha de pulsar el botón derecho del ratón en cualquier lugar.
 
 \item Tras haber seleccionado a los personajes, el programa mostrará por terminal, un listado de las posibles acciones que puede realizar. Simplemente se debe pulsar los números de aquellas acciones que se quieren realizar y seguir los pasos que indica el programa.
 \item Es importante destacar que los comportamientos que se quieran hacer, se harán para todos los objetos seleccionados. Es decir, todos los objetos seleccionados realizarán la misma acción que ha mandado el usuario.
\end{itemize}

El programa dispone de cierta funcionalidad que se puede activar/desactivar por el usuario. Esta funcionalidad es la siguiente:
\begin{itemize}
 \item \textbf{Mover la cámara de posición}. Para ello se han de usar las flechas de dirección.
 \item \textbf{Alejar/Acercar la cámara}. Para alejar la cámara se utiliza la tecla \texttt{A}, mientras que para acercar, se utiliza la tecla \texttt{Q}.
 \item \textbf{Pausar/Reaunar el juego}. Para pausar el juego, se utiliza la tecla \texttt{P}, mientras que para reanudarlo, se utiliza la tecla \texttt{O}.
 \item \textbf{Mostrar/Ocultar mapa de influcencia} encima del mapa. Para mostrar el mapa de influcencia, se utiliza la tecla \texttt{I}, mientras que para ocultarlo, se utiliza la tecla \texttt{U}.
 \item \textbf{Deshabilitar/Habilitar que haya ganador}. Por defecto, siempre hay un ganador, sin embargo, puede ser que el usuario quiera que no lo haya, por lo que para deshabilitarlo, puede utilizar la tecla \texttt{X}, mientras que para volver a habilitarlo, se utiliza la tecla \texttt{Z}.
 \item \textbf{Mostrar/Ocultar debug de los comportamientos}. Debido a que puede ser interesante visualizar los comportamientos de los personajes en el mapa, esta opción está habilitada por defecto, para ocutarlo se utiliza la tecla \texttt{N}, mientras que para mostrarlo otra vez, se utiliza la tecla \texttt{M}.
\end{itemize}
