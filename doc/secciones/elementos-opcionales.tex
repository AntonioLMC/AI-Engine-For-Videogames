%--------------------------------------------------------------------
\medskip
\section{Elementos opcionales}

En esta sección se van a explicar todos los elementos opcionales que también han sido implementados e incluidos en nuestro proyecto. \\

El primero de ellos ha sido la implementación de distintos comportamientos básico, delegados y en grupo. De hecho, hemos implementado prácticamente casi todos los comportamientos que hemos estudiado y que aparecen en las transparencias de la asignatura. La información sobre todos los comportamientos implementados se podrá encontrar en el apartado correspondiente. \\

Del mismo modo, también hemos implementado e incluido algunas estructuras de arbitraje como los \textbf{árbitros por prioridad} y los \textbf{árbitros por mezcla/por pesos} (tanto para los comportamientos acelerados como para los no acelerados). Se podrá encontrar una documentación más detallada en el apartado correspondiente. \\

Para el caso de las formaciones, se han implementado algunas otras estructuras a parte de la propia formación en circulo. Estos nuevos tipos de formación son: \textbf{formación en línea} y \textbf{formación en estrella}. Además de esto, también hemos añadido la posibilidad de crear \textbf{formaciones de formaciones} todo lo profundas que deseemos (mediante en patrón de diseño \textit{composite}). Esto se explica más detalladamente en el apartado correspondiente. \\

Uno de los elementos opcionales más importantes que hemos hecho ha sido el \textbf{modo depuración}. Durante una partida y pulsando las teclas adecuadas, se puede visualizar un montón de información sobre los movimientos que realizan los personajes, el estado de los personajes o el comportamiento táctico de los mismos. Para ello, en prácticamente todos los comportamientos implementados hemos añadido un método llamado \texttt{debug}, que será el encargado de dibujar toda la información de depuración necesaria para ese comportamiento. Además, en la clase \texttt{Character} también se han implementado ciertos métodos para mostrar, por ejemplo, el estado táctico en el que se encuentra el personaje. Por otro lado, en las formaciones también se dibujan los puntos a los deben ir los componentes de la propia formación. Para el caso de los waypoints, también contamos con los métodos necesarios para dibujar dichos puntos sobre el mapa del juego. \\

Tal y como se acaba de comentar, hemos implementado información de depuración para prácticamente todos los comportamientos (entre los que se incluye el Pathfollowing). Por tanto, como este comportamiento se usa para ir a los puntos devueltos por el pathfinding, también se podría decir que ha sido implementado el modo de depuración para el pathfinding, donde se muestran todos aquellos puntos que han sido obtenidos y por donde el personaje deberá ir. \\

Debido a cómo se han planteado las formaciones, será el propio objeto formación (el ancla) el que ejecute el pathfinding y el que vaya por los puntos que éste devuelva (los componentes de la formación simplemente se limitarán a ir a los puntos que les diga el ancla). Por tanto, para el caso de las formaciones, el pathfinding (normal y táctico) se lleva a cabo en un nivel superior y el camino obtenido se le impone a los componentes de la formación (ya que como acabo de decir, será en ancla el que lo ejecute y no cada uno de los componentes de la formación). \\

En cuanto a la información táctica implementada, no solo se ha hecho la parte de los mapas de influencia, sino que también se ha añadido un coste táctico a cada uno de los roles implementados en este proyecto (distinto coste para cada tipo de terreno). Ese coste táctico podrá repercutir directamente en la ejecución del pathfindind táctico (siempre y cuando, el personaje que lo ejecuta tenga un rol). \\

Tal y como comentaba anteriormente, debido a cómo se han planteado las formaciones, será el propio objeto formación (el ancla) el que tome ciertas decisiones y los componentes simplemente se limitarán a ir al punto que se les indique (y a acatar las decisiones tomadas por el ancla). Por tanto, ese mismo principio repercute directamente en las acciones de ataque o curación. Para el caso de las formaciones, la decisión de atacar (y también de curarse) no es tomada por los propios componentes de la formación, sino que se toma en una nivel superior (la toma el objeto tipo formación o ancla y se la impone a sus componentes).



