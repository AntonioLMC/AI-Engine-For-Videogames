%--------------------------------------------------------------------
\medskip
\section{Interacción con el usuario}
\label{interaccion}
Todo lo referente a esta sección, se encuentra dentro del paquete \texttt{com.mygdx.iadevproject.userInteraction} del proyecto. Para la interacción del sistema con el usuario, se ha hecho uso de la interfaz \texttt{InputProcessor} proporcionada por la librería LibGDX. Esta interfaz proporciona método que se llaman cada vez que, por ejemplo, se pulsa una tecla o se clicka con el ratón en la pantalla. \\

La implementación de la interacción se ha hecho siguiendo una máquina de estados. Debido a que el usuario puede realizar ciertas acciones con los personajes seleccionados, para cada tipo de acción, se ha creado un estado y dentro de ese estado, dependiendo de la tecla pulsada, se hará una acción u otra. El por qué se ha realizado de esta manera es porque hay ciertos comportamientos que requieren que el usuario seleccione a otro personaje; en una situación normal, el programa se quedaría esperando a que el usuario seleccione el personaje. Sin embargo, si se hace eso, el juego se paraliza completamente, incluso el manejo de eventos de la librería, por lo que quedar a la espera de que el usuario seleccione un personaje, no es factible. \\

Con una máquina de estados, sí podemos controlar que, cuando el usuario haya realizado la acción que se espera, se pase a otro estado que la aplique, sin tener que parar la ejecución del programa. \\

Los estados que contiene esta máquina son los siguientes:
\begin{itemize}
 \item \textbf{No hay personajes seleccionados}. Refleja el estado de que el usuario no ha seleccionado ningún personaje (estado inicial).
 \item \textbf{Se acaba de seleccionar personajes}. Refleja el estado de que el usuario acaba de seleccionar personajes.
 \item \textbf{Aplicar comportamientos accelerados}.
 \item \textbf{Aplicar comportamientos delegados}.
 \item \textbf{Aplicar un pathfinding}.
 \item \textbf{Aplicar comportamientos en grupo}.
 \item \textbf{Aplicar otros comportamientos}.
 \item \textbf{Realizar una formación}.
\end{itemize}


Si alguna de las teclas que el usuario pulsa es una de las teclas que manejan el funcionamiento del programa (como puede ser parar el juego, mover la cámara, etc), la máquina no cambia de estado, pues esas teclas se manejan de manera independiente: la máquina de estados es solamente para que el usuario realice las acciones con los personajes. \\

Para encapsular los comportamientos y acciones que el usuario pueda hacer, se ha creado la clase \texttt{UserInteraction} que es la encargada de proporcionar con métodos estáticos dichas acciones. También se encarga de mostrar los mensajes con las posibles acciones que puede realizar el usuario.